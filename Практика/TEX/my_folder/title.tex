%%% Титульник ВКР / Thesis title 
%%
%% добавить лист в pdf-навигацию 
%% add to pdf navigation menu
%%
\pdfbookmark[-1]{\thesisTitle}{tit}
%%
\thispagestyle{empty}%
\makeatletter
\newgeometry{top=2cm,bottom=2cm,left=3cm,right=1cm,headsep=0cm,footskip=0cm}
\savegeometry{NoFoot}%
\makeatother


%%% Распечатать версию документа / Print document version
%%
\begin{flushright}
%	\vspace{0pt plus0.1fill}
	\boxed{\small
		\begin{tabular}{r} 
			\textbf{Пример ВКР <<SPbPU-student-thesis-template>>.} %\\ % перенос на новую строку
			\textbf{Версия от \today % \; время:  \currenttime. % время версии
			}
		\end{tabular}
	} %end boxed
%	\vspace*{-5pt} % раскомментировать, если не хватает места
	\vspace{0pt plus0.1fill} % раскоментировать, если хватает места
\end{flushright}

{\centering%
	\Ministry\\
	\SPbPU\\
	{%\bfseries %2020 - указание на изменения, которые могут быть введены в 2020 году
		\institute}
\par}%


\vspace{0pt plus1fill} %число перед fill = кратность относительно некоторого расстояния fill, кусками которого заполнены пустые места


\noindent
\begin{minipage}{\linewidth}
	\vspace{\mfloatsep} % интервал 
	\begin{tabularx}{\linewidth}{Xl}
	&Работа допущена к защите     \\
	&\HeadTitle     \\			
	&\underline{\hspace*{0.1\textheight}} \Head     \\
	&<<\underline{\hspace*{0.05\textheight}}>> \underline{\hspace*{0.1\textheight}} \thesisYear~г.  \\ 
	\end{tabularx}
	\vspace{\mfloatsep} % интервал 	
\end{minipage}


\vspace{0pt plus2fill} %


{\centering%
	\MakeUppercase{\bfseries{Выпускная квалификационная работа \thesisDegree}}%


\intervalS%

	\MakeUppercase{\bfseries{\thesisTitle}}%

\intervalS%
	по направлению \thesisSpecialtyCodeAndTitle{}\\%
	по образовательной программе\\%
	\thesisOPCodeAndTitle
\par}%





\vspace{4mm plus2fill}%

\noindent
\begin{tabularx}{\linewidth}{lXl}
	Выполнил              &	   &             \\
	студент гр.\group     &    & \Author     \\[\mfloatsep]

	Руководитель 		  &    &             \\
	\SupervisorDegree\footnote{Должность указывают сокращенно, подразделения --- аббревиатурами. <<СПбПУ>> и аббревиатуры институтов не добавляют.} 	  &    & \Supervisor \\[\mfloatsep]
	
	Консультант		  &    & 			 \\
	по \ldots\footnote{Оформляется по решению руководителя ОП или подразделения. Поясняющие 1-3 слова помещаются на титул и в задание. <<Научный консультант>> должен иметь степень. Без печати и заверения подписи.} &    &	 \\
	\ConsultantExtraDegree 	  &    & \ConsultantExtra\\[\mfloatsep]
	
	Консультант  &    &  \\   	
	по нормоконтролю\footnote{Обязателен, из числа ППС по решению руководителя ОП или подразделения. Должность и степень не указываются. Сведения помещаются в последнюю строчку по порядку. Рецензенты не указываются.}  		 	  &    & \ConsultantNorm  % обязателен
\end{tabularx} %


%
\vspace{0pt plus4fill}% 


\begin{center}%
Санкт-Петербург\\
\thesisYear
\end{center}%
\restoregeometry
\newpage