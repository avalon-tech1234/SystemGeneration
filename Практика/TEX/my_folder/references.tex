%%% Не мянять - Do not modify
%%
%%
\clearpage                                  % В том числе гарантирует, что список литературы в оглавлении будет с правильным номером страницы
%\hypersetup{ urlcolor=black }               % Ссылки делаем чёрными
%\providecommand*{\BibDash}{}                % В стилях ugost2008 отключаем использование тире как разделителя 
\urlstyle{rm}                               % ссылки URL обычным шрифтом
\ifdefmacro{\microtypesetup}{\microtypesetup{protrusion=false}}{} % не рекомендуется применять пакет микротипографики к автоматически генерируемому списку литературы
%\newcommand{\fullbibtitle}{Список литературы} % (ГОСТ Р 7.0.11-2011, 4)
%\insertbibliofull  
%\noindent
%\begin{group}
\chapter*{Список использованных источников}	
\label{references}
\addcontentsline{toc}{chapter}{Список использованных источников}	% в оглавление 
%\printbibliography[env=SSTfirst]                         % Подключаем Bib-базы
%\ifdefmacro{\microtypesetup}{\microtypesetup{protrusion=true}}{}
%\urlstyle{tt}                               % возвращаем установки шрифта ссылок URL
%\hypersetup{ urlcolor={urlcolor} }          % Восстанавливаем цвет ссылок

\begin{enumerate}[1.]
	\item Атья М., Макдональд И. Введение в коммутативную алгебру –- М.: Мир, 1972.
	\item Винберг Э. Б. Курс алгебры –- М.: Факториал пресс, 2001.
	\item Лидл Р., Нидеррайтер Г. Конечные поля –- М.: Мир, 1988.
	\item Страуструп Б. Язык программирования С++ –- Бином, 2017.
\end{enumerate}

%\urlstyle{rm}                               % ссылки URL обычным шрифтом
%\ifdefmacro{\microtypesetup}{\microtypesetup{protrusion=false}}{} % не рекомендуется применять пакет микротипографики к автоматически генерируемому списку литературы
%\insertbibliofull                           % Подключаем Bib-базы
%\ifdefmacro{\microtypesetup}{\microtypesetup{protrusion=true}}{}
%\urlstyle{tt}                               % возвращаем установки шрифта ссылок URL
