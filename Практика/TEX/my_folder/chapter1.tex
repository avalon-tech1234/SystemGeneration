\chapter{Изучение основ общей алгебры} \label{ch1}

Для понимания алгоритма и последующей его реализации, необходимо было подготовить математическую базу: освежить часть понятий, а часть узнать впервые. Далее представлены некоторые из таких понятий.


Группа - множество, на котором задана ассоциативная бинарная операция, для которой имеется нейтральный элемент, и для каждого элемента определён обратный к нему. Если операция является коммутативной, кольцо тоже называют коммутативным.


Примеры групп: целые числа и четные числа - группы по сложению, рациональные числа без нуля - группа по умножению.


Кольцо - множество, на котором заданы две бинарные операции + и *, называемые сложением и умножением. При этом сложение коммутативно, ассоциативно и имеет нейтральный элемент, и для каждого элемента есть противоположный элемент. Умножение ассоциативно; также должна присутствовать двусторонняя дистрибутивность умножения относительно сложения. Кольцо может обладать нейтральным элементом по умножению (в этом случае оно называется кольцом с единицей) и коммутативностью умножения (в этом случае оно называется коммутативным).


Примеры колец: вещественные числа, комплексные числа, множество функций, стремящихся к нулю в единице.


Поле - множество, на котором заданы две бинарные операции + и *, называемые сложением и умножением. При этом сложение коммутативно, ассоциативно и имеет нейтральный элемент, и для каждого элемента есть противоположный элемент. Умножение ассоциативно, коммутативно, имеет нейтральный элемент, и для каждого элемента есть противоположный элемент; также должна присутствовать двусторонняя дистрибутивность умножения относительно сложения. 


Примеры полей: рациональные числа, комплексные числа.


Другими словами, кольцо является коммутативной группой по сложению, а поле является коммутативным кольцом с единицей. В группе можно складывать и вычитать элементы, кольцо добавляет операцию умножения, а в поле можно еще и делить (делением называют взятие элемента, обратного по умножению).


Конечным полем, или полем Галуа, называют поле, состоящее из конечного числа элементов. Можно показать, что количество элементов конечного поля является степенью некоторого простого числа. Это простое число называется характеристикой поля, а количество элементов поля называют порядком.


Рассмотрим пример конечного поля из двух элементов. Оно обозначается \(F_2\) или \(GF(2)\). Элементы можно определить как 0 и 1, в этом случае операции + и * определяются как сложение по модулю 2 и умножение соответственно. Также элементы этого конечного поля можно определить как «Ложь» и «Истина», тогда + и * определяются как «исключающее или» и «и» соответственно. Тем не менее, это всего лишь два представления одного и того же поля:

\begin{table} [htbp]
		\centering\small
\caption{Операции над элементами поля \(GF(2)\), представленными в виде чисел}%
	\label{tab:ToyCompare}		
\begin{tabular}{|l|l|l|}
	\hline
	$+$&$0$&$1$\\
	\hline
	$0$&0&1\\
	$1$&1&0\\ \hline
\end{tabular}
\begin{tabular}{|l|l|l|}
	\hline
	$*$&$0$&$1$\\
	\hline
	$0$&0&0\\
	$1$&0&1\\ \hline
\end{tabular}
	\normalsize% возвращаем шрифт к нормальному
\end{table}


\begin{table} [htbp]
	\centering\small
	\caption{Операции над элементами поля \(GF(2)\), представленными в виде логических объектов}%
	\label{tab:ToyCompare}		
	\begin{tabular}{|l|l|l|}
		\hline
		$+$&F&T\\
		\hline
		F&F&T\\
		T&T&F\\ \hline
	\end{tabular}
	\begin{tabular}{|l|l|l|}
		\hline
		$*$&F&T\\
		\hline
		F&F&F\\
		T&F&T\\ \hline
	\end{tabular}
	\normalsize% возвращаем шрифт к нормальному
\end{table}


Многочленом над полем называется многочлен, коэффициенты которого принадлежат заданному полю. Так как многочлены можно складывать, вычитать и умножать (и эти операции коммутативны), множество всех многочленов над данным полем является кольцом.

Аффинное преобразование – преобразование вида \(f(x) = Mx + v\), где \(M\) – обратимая матрица, \(v\) - вектор.

\newpage

%\begin{equation}
%\label{eq:Pi-ch1} % eq - equations, далее название, %ch поставлено для избежания дублирования
%\pi \approx 3,141.
%\end{equation}


%\begin{equation} 
%\label{eq:fConcept-order-ch1}
%\begin{multlined}
%(A_1,B_1)\leq (A_2,B_2)\; \Leftrightarrow \\  %\Leftrightarrow\; A_1\subseteq A_2\; %\Leftrightarrow \\ \Leftrightarrow\; B_2\subseteq B_1. 
%\end{multlined}
%\end{equation}


%% Вспомогательные команды - Additional commands
%
%\newpage % принудительное начало с новой страницы, использовать только в конце раздела
%\clearpage % осуществляется пакетом <<placeins>> в пределах секций
%\newpage\leavevmode\thispagestyle{empty}\newpage % 100 % начало новой страницы