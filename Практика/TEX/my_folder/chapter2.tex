\chapter{Разработка приложения для генерации нормализованных систем уравнений} \label{ch2}
	
\section{Общие сведения} \label{ch2:intro}

Приложение имеет консольный интерфейс на английском языке. Сгенерированные данные записываются в файлы (подробнее описано ниже). При запуске приложения можно ввести ключ \(/h\) для вызова справки или задать аргументы. Аргументы должны вводиться в следующем порядке:
\begin{enumerate}
	\item Первый – количество уравнений в генерируемой системе (натуральное число). Обязателен.
	\item Второй аргумент, если задан – ключ запуска. Допускаются ключи:
	\begin{enumerate}
		\item \(/s\) –- тихий запуск, в этом режиме в папку записываются только файлы с системой, её решением и нормализованной системой;
		\item \(/r\) –- стандартный запуск (по умолчанию), в этом режиме в папку записываются те же файлы, что в тихом режиме, а также исходные данные и все промежуточные результаты;
		\item \(/t\) –- запуск в режиме тестирования, в этом случае осуществляется стандартный запуск, после чего происходит тестирование для различных векторов. Процесс тестирования описан ниже.
	\end{enumerate} 
	\item Третий аргумент, если задан –- имя папки, куда будут записаны сгенерированные файлы. Если директория с таким именем не существует, она будет предварительно создана. По умолчанию используется имя “results”.
\end{enumerate} 

В папке, имя которой передается третьим аргументом, создается папка с именем вида YYYY.MM.DD\_HH.MM.SS –- определяется системной датой и временем. В эту папку записываются сгенерированные данные в следующих файлах:
\begin{enumerate}
	\item Случайно сгенерированные исходные данные:
	\begin{enumerate}
		\item pre\_rand/M1.txt – матрица \(M_1\);
		\item pre\_rand/M2.txt – матрица \(M_2\);
		\item pre\_rand/v1.txt – вектор \(v_1\);
		\item pre\_rand/v2.txt – вектор \(v_2\).
	\end{enumerate} 
	\item Промежуточные данные:
	\begin{enumerate}
		\item pre\_gen/S.txt – преобразование \(S\);
		\item pre\_gen/T.txt – преобразование \(T\);
		\item pre\_gen/F.txt – преобразование \(F\);
		\item inv/invM1.txt – матрица, обратная к  \(M_1\);
		\item inv/invM2.txt – матрица, обратная к  \(M_2\);
		\item inv/invF.txt – преобразование, обратное к  \(F\);
	\end{enumerate} 
	\item Результат работы программы:
	\begin{enumerate}
		\item P.txt – ненормализованная система уравнений;		\item P\_sol.txt – решение ненормализованной системы уравнений;
		\item P\_norm.txt – нормализованная система уравнений;
	\end{enumerate} 
\end{enumerate} 



%%%%
%%		
%%  \input{...} commands are used only to sychronize some parts of the text with the author guide. Authors are free to type the text directly in .tex-files   
%%  \input{...} комманды используются только, чтобы синхронизировать части текта с рекомендациями авторам. Авторы  вольны вносить текст непосредственно в файл главы  
%%  
% \input{my_folder/tex/eq-Galois} % пример двух выравнивания двух формул в окружении align


%На \firef{fig:spbpu-new-bld-autumn-ch2} приведёна фотография Нового научно-исследовательского корпуса СПбПУ.

%	\begin{figure}[ht] 
%	\center
%	\includegraphics [scale=0.27] {my_folder/images/spbpu_new_bld_autumn}
%	\caption{Новый научно-исследовательский корпус СПбПУ \cite{spbpu-gallery}} 
%	\label{fig:spbpu-new-bld-autumn-ch2}  
%	\end{figure}
	
\section{Детали реализации} \label{ch2:sec-abbr} %название по-русски
	
Наиболее важные элементы реализации приложения описаны далее в этом разделе. 	

\subsection{Служебные модули} \label{ch2:subsec-title-abbr} %название по-русски

Реализация базируется на следующих служебных модулях:
\begin{enumerate}
	\item \(Utility\) -- содержит используемые в других модулях строковые функции.
	\item \(File\_system\) -- содержит функции, используемые при работе с файловой системой.
	\item \(BOOL\) -- псевдоним для типа данных \(int\), также определены константы  \(FALSE = 0\)  и  \(TRUE = 1\).  Он используется в качестве замены логического типа данных, что позволяет достичь увеличения скорости работы с данными на 30-50\%.
\end{enumerate} 

			
\subsection{Матрицы} \label{ch2:subsec-title-abbr} %название по-русски

В пространстве имен \(matrixes\) описаны следующие классы:
\begin{enumerate}
	\item \(Row\) -- описывает строку матрицы или вектор-столбец. Агрегирует объект типа \(std::vector<BOOL>\).
	\item \(Matrix\) -- описывает квадратную матрицу, агрегирует объект типа \(std::vector<Row<BOOL> *>\). Использование указателей позволяет заметно снизить накладные расходы. Реализован метод \(init\_zeros()\), который задает размерность матрицы и инициализирует ее нулями, и метод \(initInverse()\), инициализирующий матрицу как матрицу, обратную по отношению к переданной по ссылке в качестве параметра.
	\item \(MatrixBuilder\) -- реализует паттерн Строитель, предоставляет удобный интерфейс для определения объектов \(Matrix\).
\end{enumerate} 


\subsection{Полиномы} \label{ch2:subsec-title-abbr} %название по-русски

В пространстве имен \(polynomials\) описаны следующие классы:
\begin{enumerate}
	\item \(Monomial\) -- представляет моном, или же терм. Так как все конструкции находятся в поле \(GF(2)\), степени всех переменных не превышают первую, поэтому хранится только список переменных, представленных в терме (в виде вектора). Так как полиномы генерируются самим приложением, становится возможной дальнейшая оптимизация: хранить только номера переменных, а их названия определять простым добавлением номера к букве х. Так, переменная x5 имеет номер 5. Все коэффициенты также равны единице, поэтому тоже не хранятся. Основные методы –- \(simplify()\), вызываемый после каждого изменения структуры терма, в том числе в конце работы конструктора, он гарантирует упорядоченность переменных по возрастанию; и метод \(substitute()\), подставляющий набор значений переменных в моном и возвращающий результат типа \(BOOL\).
	\item \(Polynomial\) -- описывает полином, представляет собой вектор мономов. Определены операторы \(+=\) и \(*=\), позволяющие прибавлять моном и домножать на моном, соответственно. Определен метод \(substitute()\), рекурсивно вызывающий метод \(substitute()\) каждого монома. Мономы в каждый момент отсортированы, за что отвечает метод \(simplify()\). Порядок сортировки следующий:
	\begin{enumerate}
		\item Мономы большей степени расположены раньше, чем мономы меньшей степени;
		\item Мономы равной степени сортируются лексикографически.
	\end{enumerate} 
	\item Также в пространстве имен \(polynomials\) определены классы, реализующие паттерн Строитель: \(MonomialBuilder\), упрощающий создание объектов \(Monomial\), и \(PolynomialBuilder\) и \(DNFBuiler\) – они оба упрощают создание объектов \(Polynomial\), но по-разному: \(PolynomialBuilder\) собирает полином из мономов, а \(DNFBuilder\) собирает полином как сумму других полиномов.
\end{enumerate} 


\subsection{Генерация псевдослучайных объектов} \label{ch2:subsec-title-abbr} %название по-русски

Генерация псевдослучайных объектов (ПСО) не может обеспечиваться независимым вызовом функции-генератора псевдослучайных чисел (ГПСЧ), так как вызовы могут поступать с интервалом менее одной секунды, в результате чего различные объекты будут инициализироваться одинаковыми значениями. В связи с этим создано пространство имен \(random\), содержащее несколько классов. Они генерируют случайные объекты, используя один и тот же объект ГПСЧ:
\begin{enumerate}
	\item \(RandomEngine\) -- хранит ГПСЧ \(std::mt19937\) (вихрь Мерсенна). Имеет метод \(getRandomEngine()\), возвращающий константную ссылку на этот объект. Используется для инициализации ГПСЧ других классов этого пространства имен.
	\item \(RandomMatrixFactory\) -- реализует паттерн Фабрика, генерирует ПСО типа \(Matrix\) и \(Row\).
	\item \(RandomPolynomialFactory\) -- реализует паттерн Фабрика, генерирует ПСО типа \(Polynomial\) (полином второй степени).
\end{enumerate} 


\subsection{Преобразования} \label{ch2:subsec-title-abbr} %название по-русски

В пространстве имен \(transformations\) определены следующие классы:
\begin{enumerate}
	\item \(Transformation\) –- определяет преобразование. Инкапсулирует \(std::vector<Polynomial>\), каждый полином соответствует преобразованию одной координаты. В нем определен метод \(initComposition()\), инициализирующий преобразование как композицию двух преобразований, передаваемых по константной ссылке. Также определен метод \(substitute()\), вызывающий метод \(substitute()\) для полиномов всех координат, и возвращающий \(std::vector<BOOL>\) (по ссылке, чтобы избежать лишнего копирования), и метод \(normalize()\), нормализующий систему (алгоритм его работы представлен в соответствующем разделе ниже).
	\item \(AffineTransformation\) –- наследуется от класса \(Transformation\) и позволяет определять аффинное преобразование по матрице \(M\) и вектору \(v\) как \(F(x) = Mx + v\).
	\item \(TransformationBuilder\) –- реализует паттерн строитель для объектов \(Transformation\).
\end{enumerate} 


\subsection{Ввод-вывод} \label{ch2:subsec-title-abbr} %название по-русски

Для организации единообразной структуры ввода-вывода в проекте в пространстве имен \(IO\) определены следующие классы:
\begin{enumerate}
	\item \(Writer\) –- производит запись в файл, инкапсулирует объект типа \(ofstream\). Может записывать в файл объекты типов \(Row\), \(Matrix\), \(Polynomial\).
	\item \(Reader\) –- производит чтение из файла, инкапсулирует объект типа \(ifstream\). Может принимать из файла объекты типов \(Row\), \(Matrix\), \(Transformation\).
	\item \(Parser\) –- используется при чтении \(Transformation\) из файла, разбирает строку в объект \(Polynomial\).
	\item \(ParserBackground\) –- используется классом \(Parser\), предоставляет низкоуровневый функционал для разбора строк. В том числе, содержит машину состояний.
\end{enumerate} 


\subsection{Высокоуровневый алгоритм и взаимодействие с пользователем} \label{ch2:subsec-title-abbr} %название по-русски

Ввиду несложности взаимодействия с пользователем, оно определено прямо в файле \(Main\), содержащем точку входа приложения. Метод \(main()\) принимает параметры (что позволяет передавать аргументы прямо из командной строки), если же они не поступили, то аргументы запрашиваются, если же они снова не поступают, то показывается окно справки. Разбор аргументов осуществляется также в файле \(Main\).

После приема параметров создается объект класса \(Environment\) и запускается его метод \(run()\) с параметрами (в режиме тестирования или нет, удалить лишние файлы в конце работы приложения или нет, печатать аргументы в консоль или нет – последний аргумент во всех случаях \(true\), но легко может быть изменен при добавлении новых режимов запуска). В методе \(run()\), в зависимости от переданных параметров, запускаются методы \(generateSystem()\), \(solveSystem()\), \(normalizeSystem()\), \(testYourself()\), реализующими, соответственно, генерацию системы, решение системы, нормализацию систему и тестирования решения системы. Код этих методов приведен в соответствующем приложении.











\subsubsection{Название подподпараграфа} \label{ch2:subsubsec-title-abbr} %название по-русски
	
	
Название подпараграфа оформляется с помощью команды  \texttt{\textbackslash{}subsubsecti\-on\{...\}}.



\input{my_folder/tex/enumeration} % правила использования перечислений	

	
Оформление псевдокода необходимо осуществлять с помощью пакета \verb|algorithm2e| в окружении \verb|algorithm|. Данное окружение интерпретируется в шаблоне как рисунок. Пример оформления псевдокода алгоритма приведён на \firef{alg:AlgoFDSCALING}. 
	
	
\input{my_folder/tex/pseudocode-agl-DTestsFDScaling} % пример оформления псевдокода алгоритма 	

	
	\section{Название параграфа} \label{ch2:sec-very-short-title} %название по-русски


	
\input{my_folder/tex/eq-equation-multilined} % пример оформления одиночной формулы в несколько строк

\input{my_folder/tex/fig-spbpu-sc-four-in-one} % пример подключения 4х иллюстраций в одном рисунке

%\input{my_folder/tex/fig-spbpu-whitehall-three-in-one} % пример подключения 3х иллюстрации в одном рисунке
%
%\input{my_folder/tex/fig-spbpu-main-bld-two-in-one} % пример подключения 2х иллюстраций в одном рисунке

\input{my_folder/tex/tab-more-than-one-page} % пример подключения таблицы на несколько страциц


\begin{table} [htbp]% Пример оформления таблицы
	\centering\small
	\caption{Пример представления данных для сквозного примера по ВКР \cite{Peskov2004}}%
	\label{tab:ToyCompare}		
		\begin{tabular}{|l|l|l|l|l|l|}
			\hline
			$G$&$m_1$&$m_2$&$m_3$&$m_4$&$K$\\
			\hline
			$g_1$&0&1&1&0&1\\ \hline
			$g_2$&1&2&0&1&1\\ \hline
			$g_3$&0&1&0&1&1\\ \hline
			$g_4$&1&2&1&0&2\\ \hline
			$g_5$&1&1&0&1&2\\ \hline
			$g_6$&1&1&1&2&2\\ \hline		
		\end{tabular}
%	\caption*{\raggedright\hspace*{2.5em} Составлено (или/и рассчитано) по \cite{Peskov2004}} %Если проведена авторская обработка или расчеты по какому-либо источнику	
	\normalsize% возвращаем шрифт к нормальному
\end{table}



%% please, before using, read the author guide carefully

\input{my_folder/tex/tab-toy-context-minipage} % пример подключения minipage

\input{my_folder/tex/fig-spbpu-new-bld-autumn-minipage} % пример подключения minipage




\input{my_folder/tex/rules-theorem-like-expressions} 

По аналогии с нумерацией формул, рисунков и таблиц нумеруются и иные текстово-графические объекты, то есть включаем в нумерацию номер главы, например: теорема 3.1. для первой теоремы третьей главы монографии. Команды \LaTeX{} выставляют нумерацию и форматирование автоматически. Полный перечень команд для подготовки текстово-графических и иных объектов находится в подробных методических рекомендациях \cite{spbpu-bci-template-author-guide}. 


\input{my_folder/tex/rules-list-of-environments} % список некоторых окружений


\input{my_folder/tex/theorem-example} %пример оформления теоремы


\input{my_folder/tex/definition-example} %пример оформления определения


Вместо теоремо-подобных окружений для вставки небольших текстово-графических объектов иногда используются команды. Типичным примером такого подхода является команда \verb|\footnote{text}|\footnote{Внимание! Команда вставляется непосредственно после слова, куда вставляется сноска (без пробела). Лишние пробелы также не указываются внутри команды перед и после фигурных скобок.}, где в аргументе \verb|text| указывают текст \textit{подстрочной ссылки (сноски)}.В них \textit{нельзя добавлять веб-ссылки или цитировать литературу}. Для этих целей используется список литературы. Нумерация сносок сквозная по ВКР без точки на конце выставляется в шаблоне автоматически, однако в каждом приложении к ВКР нумерация, зависящая от номера приложения, выставляется префикс <<П>>, например <<П1.1>> --- первая сноска первого приложения. 




%\FloatBarrier % заставить рисунки и другие подвижные (float) элементы остановиться


\section{Выводы} \label{ch2:conclusion}

Текст заключения ко второй главе. Пример ссылок \cite{Article,Book,Booklet,Conference,Inbook,Incollection,Manual,Mastersthesis,Misc,Phdthesis,Proceedings,Techreport,Unpublished,badiou:briefings}, а также ссылок с указанием страниц, на котором отображены те или иные текстово-графические объекты  \cite[с.~96]{Naidenova2017} или в виде мультицитаты на несколько источников \cites[с.~96]{Naidenova2017}[с.~46]{Ganter1999}. Часть библиографических записей носит иллюстративный характер и не имеет отношения к реальной литературе. 

Короткое имя каждого библиографического источника содержится в специальном файле \verb|my_biblio.bib|, расположенном в папке \verb|my_folder|. Там же находятся исходные данные, которые с помощью программы \texttt{Biber} и стилевого файла \texttt{Biblatex-GOST} \cite{ctan-biblatex-gost} приведены в списке использованных источников согласно ГОСТ 7.0.5-2008.
Многообразные реальные примеры исходных библиографических данных можно посмотреть по ссылке \cite{ctan-biblatex-gost-examples}.

Как правило, ВКР должна состоять из четырех глав. Оставшиеся главы можно создать по образцу первых двух и подключить с помощью команды \verb|\input| к исходному коду ВКР. Далее в приложении \ref{appendix-MikTeX-TexStudio} приведены краткие инструкции запуска исходного кода ВКР \cite{latex-miktex,latex-texstudio}.

В приложении \ref{appendix-extra-examples} приведено подключение некоторых текстово-графических объектов. Они оформляются по приведенным ранее правилам. В качестве номера структурного элемента вместо номера главы используется <<П>> с номером главы. Текстово-графические объекты из приложений не учитываются в реферате.



%% Вспомогательные команды - Additional commands
%
%\newpage % принудительное начало с новой страницы, использовать только в конце раздела
%\clearpage % осуществляется пакетом <<placeins>> в пределах секций
%\newpage\leavevmode\thispagestyle{empty}\newpage % 100 % начало новой страницы