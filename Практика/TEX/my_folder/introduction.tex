\chapter*{Введение} % * не проставляет номер
\addcontentsline{toc}{chapter}{Введение} % вносим в содержание

Криптография --- это наука о методах обеспечения конфиденциальности и целостности данных. В настоящее время она является одной из важнейших областей дискретной математики. Методы криптографии применяются практически во всех отраслях, требующих обеспечения безопасности данных: электронная коммерция, технологии криптовалюты, электронный документооборот, телекоммуникации.


Одним из наиболее популярных направлений криптографии является криптография с открытым ключом. Этот принцип предусматривает наличие двух ключей: публичного (открытого), используемого для шифрования данных, и секретного (закрытого), используемого для расшифровки. При этом к секретному ключу предъявляется требование невозможности его вычисления за разумный срок.


В рамках данной практики поставлена цель разработать приложение, реализующее часть криптографической системы с открытым ключом. В ней в качестве открытого ключа выступает система уравнений, вычисляемая на основе случайно сгенерированных исходных данных. Процесс шифрования состоит в подстановке вектора переменных в неё, а процесс расшифровки – в решении системы. Расшифровка не может быть быстро произведена без знания исходных данных для системы уравнений, что и обеспечивает криптостойкость разрабатываемой системы.


Для достижения выбранной цели поставлены следующие задачи:

\begin{enumerate}
	\item Изучить основы общей алгебры (и других математических инструментов, необходимых для понимания и реализации используемых алгоритмов).
	\item Разработать модуль генерации систем уравнений на основе случайно генерируемых входных данных (матриц и векторов).
	\item Разработать модуль решения систем уравнений.
	\item Разработать модуль нормализации системы уравнений (представления в виде большего количества более простых уравнений).
\end{enumerate} 


Разработанное приложение имеет консольный интерфейс и написано на языке C++, использовался стандарт ISO C++14. Разработка велась в среде Microsoft Visual Studio 2017. В процессе разработки использовалась система контроля версий Git.

%% Вспомогательные команды - Additional commands
%\newpage % принудительное начало с новой страницы, использовать только в конце раздела
%\clearpage % осуществляется пакетом <<placeins>> в пределах секций
%\newpage\leavevmode\thispagestyle{empty}\newpage % 100 % начало новой строки